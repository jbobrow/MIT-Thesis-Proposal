% $Log: abstract.tex,v $
% Revision 1.1  93/05/14  14:56:25  starflt
% Initial revision
% 
% Revision 1.1  90/05/04  10:41:01  lwvanels
% Initial revision
% 
%
%% The text of your abstract and nothing else (other than comments) goes here.
%% It will be single-spaced and the rest of the text that is supposed to go on
%% the abstract page will be generated by the abstractpage environment.  This
%% file should be \input (not \include 'd) from cover.tex.
For this thesis, I have developed a framework based on previous work in the field of cellular automata, simulation of cellular systems that progress in discrete time through interactions with their coincident neighbors. In an effort to make systems thinking more accessible, I will design and develop a set of tangible cellular automata, each capable of updating itself in discrete time and deciding its next state based on its neighbors. This Tangible User Interface(TUI) will be compared and contrasted with a Graphical User Interface(GUI) or software simulation to measure the benefits of tangible interaction.

Just as John Conway sparked much interest and investigation through cellular automata with his ruleset for "Game of Life," I aim to bring this kind of investigation to a new audience, one that doesn't consider itself systems thinkers. The original cellular automata studies were conducted in the 1960s on graph paper, blackboards, and table top board games such as Go. AutomaTiles seeks to make explorers out of a new generation. If successful, AutomaTiles will prove to be as endlessly explorable as Legos, but with a focus on the in between, how the interaction between two or more components could build to arrive at something as complex as physics, life, or social systems.
