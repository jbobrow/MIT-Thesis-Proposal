%% This is an example first chapter.  You should put chapter/appendix that you
%% write into a separate file, and add a line \include{yourfilename} to
%% main.tex, where `yourfilename.tex' is the name of the chapter/appendix file.
%% You can process specific files by typing their names in at the 
%% \files=
%% prompt when you run the file main.tex through LaTeX.
\chapter{Proposal for Playful Engagement with Systems Thinking}
\begin{quotation}"Children must be taught how to think, not what to think."\end{quotation}
 — Margaret Mead, cultural anthropologist 

\begin{quotation}"We shape our tools and then our tools shape us.”\end{quotation}
 — Father John Culkin

\section{Introduction}
The tools and toys that allow us to explore and learn in a space define how we think about any given space. Games have long been a means for adopting new rules and entering a “magic circle” to not simply think about, but engage with different kinds of systems. It is precisely this curiosity and willingness to explore that become available during times of play that I seek to engage people with an important and new way of thinking, systems thinking. Prior to the 1950s, much of science focussed on discrete components, trying to understand every detail isolated on its own. Around 1940, Jay Forrester formed the MIT System Dynamics group, and while this was far from the birth of systems thinking, it provided a concentrated effort to define precisely what that would look like. Systems thinking is linked to all sorts of beneficial behaviors such as long term planning, being a team player, or empathy. So how do we engage more people to become systems thinkers. In 1970, a mathematician named John Conway created a game for no players, a deceptively simple simulation or cellular automata called Game of Life. A Scientific American article captured the imaginations of other like minded individuals and brought about deep exploration into this fictional world, where everything could potentially be known, yet the results never cease to surprise us. The kind of people that engage with Game of Life are largely the kind of people that already think about systems. To help bring this kind of play with systems in general to a broader audience, I seek to lower the barrier of entry to playing with cellular automata and furthermore systems thinking in general.

\section{Great Toys}

Golan Levin once noted, "A great tool is instantly knowable and infinitely masterable.” Golan was referring to tools like the pencil or piano, but this rule can apply directly to toys or games as well. In an effort to make a tool instantly knowable, I have taken a once graph paper calculation or screen based simulation into the realm of the physical, and designed a set of tangible cellular automata, table top toys which rely on their connections rather than standing out on their own. If Lego is a great tool for exploring static systems, AutomaTiles will be a great tool for playing with dynamic systems, or systems tinkering. Recent advances in technology such as fast, small and affordable embedded devices have made the area of Distributed Computational Toys(DCTs) a possibility.

\section{Related Work}

Of course DCTs are not that new, there is existing work in the field. Projects such as Leah Buechley’s Boda Blocks, explores tangible 3Dimensional cellular automata, with a GUI for programming homogeneous or heterogeneous rulesets. So why are AutomaTiles different? Just as Lego Mindstorms provided an entry way into computational tinkering and Little Bits seeks to reach a broader and less technical audience, AutomaTiles are designed to engage anyone regardless of age or prior knowledge. A few specific areas where AutomaTiles truly innovate are in form factor, ease of use, customization, as well as personalization. The form factor of AutomaTiles is a regular hexagon, which provides the highest amount of shared sides allowing for dynamics based on 6 neighbors. To accomplish communication discreetly and with durability and robustness considered, AutomaTiles utilize infrared LEDs and phototransistors which happily transmit through plastic or silicone enclosures. As learned from playing games like Settlers of Catan or Hive, a chess-like boardgame with an emergent board, singular pieces in a grid do not like to stay put while actively moving pieces around. In similar fashion to Little Bits, each side contains two magnets of opposing polarity, providing a satisfying feeling of self correction in the devices themself. Simply placing AutomaTiles on a table together will result in a well connected lattice. 
%%\subsection{Six Forty by Four Eighty}
%%\subsection{Sifteo}
%%\subsection{Little Bits}
\subsection{Customization}
When speaking of customization, there are few areas in which AutomaTiles will thrive. Firstly, the physical form factor of AutomaTiles is designed to include a skin layer, which can serve as a cool custom layer, or provide narrative cues about the piece and its potential behavior. By taking advantage of the RGB LED, a skin can contain a two tone graphic and mask one color with lighting to create the illusion of a switchable surface. An example of this could change an AutomaTile’s emotion from happy to angry. This is a great advantage over having to map the relationship of a color to a specific state or emotion. Different skins can provide quick legibility to systems of all types. In addition to physical traits, the AutomaTiles are quickly reprogrammable. One might say they like to gossip. Using their infrared transmitters, AutomaTiles can quickly send a signal for neighbors to enter programming mode where they await their new set of rules. This process moves updating rulesets away from a computer, keeping the interaction tactile and engaging. The possible rulesets are defined by the firmware, but potential expansion could allow transmission of scripting for a much larger domain of easily programmable behaviors.

\subsection{Evaluation}

AutomaTiles exists to engage a population in a subject matter unfamiliar or otherwise less accessible to them. To understand if AutomaTiles are a success, there are two clear metrics, engagement and learning. Engagement should be measured by observation as well as self reporting, hopefully there is a consistency between the two, but the redundancy should help to provide a more objective viewpoint. Secondly, learning can be measured by providing a survey of traits before playing with AutomaTiles and once again after an hour of exposure. Organizations such as the Waters Foundation, which launched initiatives to bring systems thinking to schools, have existing survey questions that rely on self reporting of ability to utilize the kind of skills systems thinking might provide. A measure of success might be the engagement users exhibit, or some insights about potential to improve engagement. Similarly, the learning goals are largely about providing a new perspective, a new lens for the world that users might not apply immediately, but through exposure, might be more likely to think through. Louis Kahn once asked the question, and I paraphrase, what does a brick want to be? For me, answering the question, what does an AutomaTile want to be with? is tantamount to success.

A secondary goal for AutomaTiles is to become a platform for playing with and exploring emergent systems. Creating a platform that another group can work with, for example, programming a new ruleset would be a great success for the extensibility of a relatively generalizable tangible cellular automata. Since the physicality of AutomaTiles affords specific interactions, use cases where location and rearrangement of a lattice or network could be most beneficial. Similarly, the ability to propagate heterogeneous rulesets might lead to some useful analytical tools for discovering and understanding robust networks.

\subsection{Time Frame}

AutomaTiles is a large undertaking for a few reasons. This thesis is an exercise in Design, industrial, electrical, interaction, and user experience. While much of the heavy lifting of designing hardware and firmware to beta test prototypes was accomplished through the Summer and Fall semesters, there is plenty of work still to be done. Having hands on devices is crucial to testing whether AutomaTiles is a worthwhile toy to address the issue of raising systems thinkers. Revision 3 of AutomaTiles are stable enough to implement most needs through firmware and might prove robust enough to simply expand production and make a larger population of AutomaTiles available to a wider audience. The technical timeline includes making data/ruleset transmission more robust, a robust synchronized heartbeat for pacemaker operation in muted mode, design and development of a pacemaker tile as well as a programmer tile, which might be one in the same. The design timeline includes design and development towards a functional narrative, creating a world within which adults and children alike can play. Character development both visually and conceptually is important to raise levels of engagement especially with regard to repeatability. (This might be measured by simply asking if users could imagine themselves continuing to use them, or even if they want to continue playing with them). A parallel track for design includes tools for design which are building both a lexicon and taxonomy around AutomaTiles. Extending the simulation tool to begin to classify organisms or arrangements of AutomaTiles of varying sizes will guide the world that AutomaTiles and players, temporarily, will live in.

Continuing in November and finishing by the end of November will be the development of a robust ruleset transmission. Documentation for transmitting these rules and a device for doing so will follow shortly after. Prototyping of the pacemaker and programmer will resume in December with goals of final board design and enclosure to be fabricated in January. The month of January will provide time for testing by advanced users programming through the API, i.e. alternate rulesets and prepare for deployment with one or more participating groups, not limited to a gaming lab from Northeastern or the Wyss Institute for exploring cooperative robots.
The Spring semester will largely be focussing on deploying the framework as well as tools to explore play with AutomaTiles. I plan to organize events and curricula with schools to directly engage students with systems thinking. Collaborations in the space of narrative creation will also be a focus, including new skins and systems to be played with. A successful AutomaTiles will have at least 2 rulesets playable with clear distinction between the worlds created.

%%\section{Motivation}
%%\section{Related Work}
%%\section{Proposed Work}
%%\section{Originality and Contribution}
%%\section{Evaluation}
%%\section{Timeline}


